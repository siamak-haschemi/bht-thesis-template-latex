\chapter{FAQ}
In diesem Kapitel werden häufige Fragen beantwortet.  Bitte lies dir alle Fragen einmal durch, damit wir häufige Punkte nicht mehrfach erklären müssen und wertvolle Zeit sparen.

\paragraph{Wie viel technisches Vorwissen kann beim Leser vorausgesetzt werden?}
Der Bericht richtet sich an Informatiker, daher kann grundlegendes technisches Wissen wie Git vorausgesetzt werden. Spezifisches Fachwissen, etwa der technische Aufbau einer Browser-Extension, sollte jedoch nicht erwartet werden.

\paragraph{Welche Schriftart und Formatierung soll ich verwenden?}
Die Wahl der Schriftart ist nicht entscheidend. Für gedruckte Berichte eignen sich Serifenschriften, da sie besser lesbar sind; du kannst den Standardfont oder den BHT-Font verwenden. Abstände und genaue Formatierung spielen keine Rolle für die Bewertung – entscheidend ist allein der Inhalt. Eine zu große oder zu kleine Zeilenhöhe führt nicht zu einer schlechteren Note.

\paragraph{Sollte man den zweiten Gutachter kontaktieren?}
Es ist nicht verpflichtend, den zweiten Gutachter zu kontaktieren, und häufig erhält man keine Antwort. Dennoch ist es empfehlenswert, eine kurze Probe oder einen Auszug der Arbeit zu schicken und um Feedback zu bitten. So kann man später nachvollziehbar machen, dass man Rücksprache gesucht hat – besonders hilfreich, falls es größere Meinungsunterschiede geben sollte.

\paragraph{Wo sollte ich für die Bachelorarbeit recherchieren?}
Geeignete Recherchequellen sind Google Scholar, arXiv, Fachartikel, Journals, Webseiten sowie Bücher; auch ChatGPT Deep Research kann zur Orientierung genutzt werden. Der Anteil reiner Webseitenquellen sollte jedoch maximal ein Drittel der gesamten Literatur ausmachen. Überwiegen Webseiten deutlich, wirkt die Arbeit nicht ausreichend wissenschaftlich fundiert – darauf wird bei der Bewertung geachtet.

\paragraph{Ist der Einsatz von KI-Werkzeugen erlaubt?}
KI-Werkzeuge dürfen unterstützend eingesetzt werden (z.\,B. für Code-Generierung oder Textüberarbeitung), solange die Inhalte verstanden und verantwortet werden. Unreflektiertes Übernehmen ist nicht zulässig.

\paragraph{Welche formalen Anforderungen gelten für den Schreibstil?}
Die Arbeit wird nicht in Ich-Form verfasst. Aussagen müssen mit Quellen belegt werden. Typischerweise werden 20–30 Referenzen erwartet, wobei Webquellen nur einen begrenzten Anteil ausmachen sollten. Zudem empfiehlt es sich, das Abstract erst am Ende zu schreiben, da dort die Ergebnisse zusammengefasst werden und oft erst gegen Abschluss der Arbeit klar ist, was tatsächlich erreicht wurde.
