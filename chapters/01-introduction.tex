\chapter{Einführung}
Das Wichtigste zuerst: Gib dem GitHub-Repository\footnote{\url{https://github.com/julianbht/bht-thesis-template-latex}} einen Stern!!! 

Aber im Ernst - die Idee ist, dass jeder ein bisschen zum Repository beiträgt. Wenn du eine Verbesserung hast, einen Fehler entdeckst oder auch nur einen Rechtschreibfehler findest, erstelle bitte einen Pull Request. Ziel ist es, dass alle gemeinsam beitragen, damit die Vorlage mit der Zeit immer besser wird.

Wenn du diese Anleitung zum ersten Mal liest, solltest du folgende Schritte abarbeiten:

\begin{enumerate}
    \item Erstelle einen Zeitplan mit einigen Meilensteinen und sende ihn Siamak über Discord.\footnote{\url{TODO: Discord Link}}

    \item Lies dir das FAQ durch.

    \item Da dies ein Template Repository ist, kannst du über die GitHub-Oberfläche mit dem Button \emph{Use This Template} deine eigene Version erstellen (siehe \cref{fig:step-1}).

    \begin{figure}[H]
        \centering
        \includegraphics[width=\textwidth]{figures/step-1.png}
        \caption{Schritt 3: Template Repository verwenden}
        \label{fig:step-1}
    \end{figure}

    \item Füge Siamak zu dem Repository hinzu. Dieser importiert anschließend dein GitHub-Projekt nach Overleaf und fügt dich bei dem Overleaf Projekt hinzu (siehe \cref{fig:step-2}).

    \begin{figure}[H]
        \centering
        \includegraphics[width=\textwidth]{figures/step-2.png}
        \caption{Schritt 4: Siamak hinzufügen}
        \label{fig:step-2}
    \end{figure}

    \item Lösche den Erklärungskram und lege los!
\end{enumerate}

